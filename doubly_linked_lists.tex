\documentclass[10pt,twocolumn,letterpaper]{article}
\usepackage{amsmath}

\begin{document}
\title{Doubly Linked Lists}
\maketitle

\section{Operators}

\begin{enumerate}
\item \textit{head}
\item \textit{tail}
\item \textit{rhead}
\item \textit{rtail}
\item \textit{append}
\item \textit{prepend}
\item \textit{len}
\end{enumerate}

\section{Special Constants}
\begin{enumerate}
\item \textit{nil}
\end{enumerate}

\section{Axioms}

These have not been proven to be a full axiomatization.

$l_i$ are list constants and $e_i$ are elements.

\begin{equation}
\begin{aligned}
&len(\textit{nil}) = 0 \\
&l = nil \vee len(tail(l)) = len(l) - 1 \\
&l = nil \vee len(rtail(l)) = len(l) - 1 \\
&l_1 = l_2 \implies (head(l_1) = head(l_2) \wedge tail(l_1) = tail(l_2)) \\
&l_1 = l_2 \implies (rhead(l_1) = rhead(l_2) \wedge rtail(l_1) = rtail(l_2)) \\
&tail(rtail(l)) = rtail(tail(l)) \\
&len(l) = 1 \implies tail(l) = nil \\
&len(prepend(l, e)) = len(l) + 1 \\
&len(append(l, e)) = len(l) + 1 \\
&head(prepend(l, e)) = e  \\
&rhead(append(l, e)) = e  \\
&tail(prepend(l, e)) = l  \\
&rtail(append(l, e)) = l  \\
%%&tail(prepend(nil, e)) = nil \\
%%&rtail(prepend(nil, e)) = nil \\
%%&tail(append(nil, e)) = nil \\
%%&rtail(append(nil, e)) = nil
\end{aligned}
\end{equation}

\section{Ideas}
\begin{enumerate}
\item From Nikolaj Bjorner's thesis: Could express as a word unification problem (with concatenation as primitve operator). But, this is doubly exponential in the worst case.
\item However, I wonder if it's possible to have concatenation as a 'hidden' operator, used only in decision procedure rules. Not sure if that makes any sense. It would be nice to recognize that two connected FIFOs are equivalent to a single FIFO
\end{enumerate}
\end{document}